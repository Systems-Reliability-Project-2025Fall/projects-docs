\documentclass[conference]{IEEEtran}
\IEEEoverridecommandlockouts
% The preceding line is only needed to identify funding in the first footnote. If that is unneeded, please comment it out.
\usepackage{cite}
\usepackage{amsmath,amssymb,amsfonts}
\usepackage{algorithmic}
\usepackage{graphicx}
\usepackage{textcomp}
\usepackage{xcolor}
\usepackage[hidelinks,colorlinks=true,linkcolor=blue,citecolor=blue]{hyperref}
\def\BibTeX{{\rm B\kern-.05em{\sc i\kern-.025em b}\kern-.08em
    T\kern-.1667em\lower.7ex\hbox{E}\kern-.125emX}}
\begin{document}

\title{Project Proposal: AIOpsLab Evaluation}
\author{\IEEEauthorblockN{Kosumi Chan}
% \IEEEauthorblockA{\textit{dept. name of organization (of Aff.)} \\
\textit{The University of North Carolina at
Chapel Hill}\\
kosumi@cs.unc.edu
% City, Country \\
% }
\and
\IEEEauthorblockN{Chai}
%\IEEEauthorblockA{\textit{dept. name of organization (of Aff.)} \\
\textit{The University of North Carolina at
Chapel Hill}\\
%City, Country \\
@cs.unc.edu}

\maketitle

% \begin{abstract}
% This document is a model and instructions for \LaTeX.
% This and the IEEEtran.cls file define the components of your paper [title, text, heads, etc.]. *CRITICAL: Do Not Use Symbols, Special Characters, Footnotes, 
% or Math in Paper Title or Abstract.
% \end{abstract}

% \begin{IEEEkeywords}
% component, formatting, style, styling, insert
% \end{IEEEkeywords}

\section{Introduction}

scale 
outage
loss 

responsive 

\section{Background}

\subsection{Clouds}

\subsection{Faults and incidents}

\subsection{Compound AI Systems}
Context length limit

\subsection{}


\subsection{\textsc{AIOps}LAB}
\textsc{AIOps}LAB\cite{chen2025aiopslab} is an extensible framework for evaluation of AIOps agents. 

\subsubsection{Orchestrator}


\section{Project tasks}

\subsection{Partial Reproduction}
\subsection{Evaluate State-of-Art models}





\bibliography{refs}
\bibliographystyle{plain}

\vspace{12pt}

\end{document}
