\documentclass[conference]{IEEEtran}
\IEEEoverridecommandlockouts
% The preceding line is only needed to identify funding in the first footnote. If that is unneeded, please comment it out.
\usepackage{cite}
\usepackage{amsmath,amssymb,amsfonts}
\usepackage{algorithmic}
\usepackage{graphicx}
\usepackage{textcomp}
\usepackage{xcolor}
\usepackage[hidelinks,colorlinks=true,linkcolor=blue,citecolor=blue]{hyperref}
\def\BibTeX{{\rm B\kern-.05em{\sc i\kern-.025em b}\kern-.08em
    T\kern-.1667em\lower.7ex\hbox{E}\kern-.125emX}}
\begin{document}

\title{Project Proposal: AIOpsLab Evaluation}
\author{\IEEEauthorblockN{Kosumi Chan}
% \IEEEauthorblockA{\textit{dept. name of organization (of Aff.)} \\
\textit{The University of North Carolina at
Chapel Hill}\\
kosumi@cs.unc.edu
% City, Country \\
% }
\and
\IEEEauthorblockN{Chai}
%\IEEEauthorblockA{\textit{dept. name of organization (of Aff.)} \\
\textit{The University of North Carolina at
Chapel Hill}\\
%City, Country \\
@cs.unc.edu}

\maketitle

% \begin{abstract}
% This document is a model and instructions for \LaTeX.
% This and the IEEEtran.cls file define the components of your paper [title, text, heads, etc.]. *CRITICAL: Do Not Use Symbols, Special Characters, Footnotes, 
% or Math in Paper Title or Abstract.
% \end{abstract}

% \begin{IEEEkeywords}
% component, formatting, style, styling, insert
% \end{IEEEkeywords}

\section{Introduction}

scale 
outage
loss 

response time  


\section{Background}

\subsection{Clouds}

\subsection{Faults and incidents}

\subsection{Compound AI Systems(AI agents) and AIOps}
Ideally, given a clear problem description, an LLM should be able to return an answer in one step, as is the case for numerous simple tasks. However, many real-world tasks require multi-step reasoning for 2 reasons: 
        \begin{itemize}
                \item The initial context is usually insufficient for tackling the task so the LLM needs to get more context information in the following steps  
                \item Even if a complete context is available, a longer context increases the risk of hallucination and failure, and there is a limit on context length.
        \end{itemize}

        CloudOps is such a real-world task with insufficient initial context. Cloud faults are unpredictable in general, can be non-deterministic or latent. Human DevOps engineers or SREs(Site Reliability Engineer) detect, localize and mitigate cloud faults by observing and acting iteratively with tools. 

        AIOps automates human reasoning with LLM reasoning, but the agent still carries out the task by observing and acting with tools in multiple steps.     

        State-of-Art LLM agent architectures include ReAct\cite{yao2023reactsynergizingreasoningacting} and FLASH\cite{zhang2024flash}.

\subsection{\textsc{AIOps}LAB}
\textsc{AIOps}LAB\cite{chen2025aiopslab} is an comprehensive framework for evaluation of AIOps agents in realistic microservice environments.  

AIOpsLab integrates multiple components, including a workload generator, extensible fault library, telemetry collectors, and an orchestrator that provides a unified \textbf{Agent-Cloud Interface (ACI)}. This design enables agents to interact dynamically with cloud environments, perform tasks such as detection, localization, root-cause analysis, and mitigation, and receive structured feedback. By offering benchmark problems across diverse operational scenarios, AIOpsLab serves as a testbed for evaluating state-of-the-art LLM-based agents.

\subsubsection{Orchestrator}


\section{Project tasks}

\subsection{Partial Reproduction}
\subsection{Evaluate State-of-Art models}






\bibliography{refs}
\bibliographystyle{plain}

\vspace{12pt}

\end{document}
